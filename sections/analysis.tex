\subsection{Especificación de requisitos de software}

En esta sección se presenta la Especificación de Requisitos de Software (ERS) para la aplicación móvil que forma parte del proyecto terminal “Sistema web y móvil para gestionar ventas y acceso en eventos de asistencia masiva utilizando etiquetas NFC” según el estándar de IEEE 830.

%\subsubsection{Propósito}

%El propósito de este documento es definir la estructura y diseño de la aplicación móvil. Este documento va dirigido a toda persona interesada en el funcionamiento y estructuración del software a desarrollar.

\subsubsection{Ámbito del software}

\begin{itemize}
\item	La aplicación móvil llevará por nombre “TapTop”.
\item	La aplicación interpretará un mensaje de voz a lenguaje de señas, representando el mensaje por medio de imágenes o animaciones.
\item	La aplicación no guardará los mensajes de voz.
\item	La aplicación contará con la capacidad de realizar síntesis de voz.
\item	No se podrán enviar mensajes digitales a través de la aplicación con otros usuarios.
\item	La aplicación permite la consulta de un diccionario del LSM por categorías.
\item	La aplicación reconocerá e interpretará cierta cantidad de frases limitadas por 43 palabras que se encuentran en el banco de reconocimiento.
\item	Con el desarrollo de esta aplicación se espera que las personas que utilizan el lenguaje de señas sean capaces de comunicarse con personas que no y a su vez, las personas que no hacen uso de éste sean capaces de comunicarse con personas que sí.
\item	Un beneficio de esta aplicación es que cualquier persona que cuente con un móvil Android podrá hacer uso de ella, sin hardware adicional.
\item	El objetivo de la aplicación móvil es lograr que las personas tengan mayor oportunidad de acercamiento a tecnologías que permitan realizar la comunicación sin hacer uso de hardware adicional.
\item	Se espera que el software en un futuro pueda ser capaz de admitir una ampliación a su vocabulario a reconocer.
\end{itemize}

\subsubsection{Del sistema}

\begin{itemize}
\item	La aplicación móvil tendrá conexión al servidor que llevará a cabo el procesamiento de voz y nos entregará el resultado, la conexión mediante Internet.
\item	El sistema permitirá el uso en cualquier momento.
\item	El sistema soportará diversas peticiones al mismo tiempo, el número de peticiones está por definirse de acuerdo a los algoritmos implementados.
\end{itemize}

\subsubsection{Visión general}

En las secciones siguientes se detalla la estructura y funcionamiento de la aplicación móvil, los requisitos del dispositivo que la ejecutará, diagramas de casos de uso, actividades y un esbozo general de la aplicación.

\subsubsection{Descripción general}
\paragraph{Perspectiva del producto}\paragraph{}

La aplicación móvil será totalmente independiente a cualquier otro producto, en un futuro la unión de dos aplicaciones no queda descartada, pues existen productos desarrollados y en desarrollo que contribuyen al proceso de comunicación con el uso del LSM y esta unión se podrá dar gracias a un content provider entre ambas aplicaciones. Para este trabajo se desarrollará la aplicación móvil que realice la traducción de un mensaje de voz a lenguaje de señas, se diseñará la aplicación móvil y desarrollará el sistema que permita el procesamiento de voz.

\paragraph{Funciones del producto}\paragraph{}

La aplicación móvil contará con tres funciones principales:

\begin{itemize}
\item	Traducción de voz a LSM: Esta función básicamente graba una frase en voz y la traduce a lenguaje de señas mexicana, mostrando el resultado mediante imágenes.
\item	Reproducción en voz de cualquier texto ingresado: Se toma el texto ingresado en un campo de texto y se realiza la síntesis en voz de éste.
\item	Consulta de un diccionario de LSM: Organizado por categorías, se podrán consultar palabras del LSM y la forma en que se realizan dichas señas.
\end{itemize}

\paragraph{Características de los usuarios}\paragraph{}

La aplicación móvil va dirigida a personas con interés de comunicarse principalmente con personas sordas que utilizan el LSM, además, también va dirigido a personas que utilizan el lenguaje de señas y que tienen interés en comunicarse con personas que no lo usan, estas personas deberán tener conocimiento básico del uso de aplicaciones móviles.

\paragraph{Restricciones y limitaciones}

El uso de esta aplicación se recomienda en dispositivos móviles con las siguientes características mínimas:
\begin{itemize}
\item	Smartphone Android versión 4.4.
\item	Memoria RAM de 1 GB.
\item	Memoria interna de 8 GB con memoria externa.
\item	CPU: Quad-core 1.4 GHz
\end{itemize}

Para el servidor que realizará el procesamiento:
\begin{itemize}
\item	Windows 8 o superior.
\item	Java 8 o superior.
\item	4 GB de memoria RAM.
\item	10 GB de disco duro disponibles.
\item	Procesador de doble núcleo a 1.8 GHz.
\item	MySQL server 5.6.
\end{itemize}

La aplicación será desarrollada en Android Studio versión 2.2 bajo el lenguaje de programación Java, el procesamiento de voz se llevará a cabo en un servidor con lenguaje C.

La comunicación con el servidor se llevará a cabo mediante web service a través del protocolo TCP/IP.

La aplicación móvil no contará con sistema de seguridad ya que no se estarán manejando datos delicados ni registro en ésta.

\paragraph{Suposiciones y dependencias}\paragraph{}

La aplicación móvil está diseñada para Android, si se desea migrar a otro sistema operativo se deberán revisar los requisitos del dispositivo, así como el cambio de desempeño e interfaces con el servidor, de igual forma, si el servidor se implementa en un sistema operativo, será necesario revisar la comunicación que se llevará a cabo con la aplicación móvil.

El número de palabras y/o frases en un futuro puede crecer, así que es posible que se requiera un servidor con mayor capacidad de procesamiento para llevar a cabo el reconocimiento de palabras.

\paragraph{Requisitos futuros}\paragraph{}

Las posibles mejoras a la aplicación serían la integración de un diccionario más amplio de reconocimiento, integración de un sistema de reconocimiento de imágenes para completar la comunicación bidireccional e incluir un curso sobre lenguaje de señas.

\subsubsection{Requisitos específicos}

\paragraph{Interfaces externas}

\begin{itemize}
\item	Ie1. La aplicación móvil tendrá una interfaz sencilla que permita al usuario intuir el funcionamiento y la sección de funciones de ésta.
\item	Ie2. La comunicación con el servidor se llevará a cabo mediante un web service a través del protocolo TCP-IP.
\item	Ie3. Para la realización de la función de \textit{text-to-speech} o síntesis de voz se hará uso de la API que Google nos proporciona.
\item	Ie4. El servidor utilizará el ODBC más adecuado  para comunicarse con la base de datos.
\end{itemize}

\paragraph{Funciones}
\begin{enumerate}[label=(\alph*)]
\item	Traductor de voz a LSM

\begin{itemize}
\item	Fa1. Se contará con un botón para iniciar la grabación de la voz. (Móvil)
\item	Fa2. Se enviará la voz o las características de ésta al servidor a través de un web service. (Móvil)
\item	Fa3. Se realizará el procesamiento de voz para determinar las palabras mencionadas. (Servidor)
\item	Fa4. Se mapeará la frase detectada con la estructura correspondiente al LSM. (Servidor)
\item	Fa5. Se leerá de la base de datos las imágenes o animaciones correspondientes a la frase detectada. (Servidor)
\item	Fa6. Se enviará a la aplicación móvil las imágenes o códigos de éstas para que en la pantalla muestre el mensaje con la estructura correcta. (Servidor)
\item	Fa7. La aplicación móvil mostrará en pantalla las imágenes o animaciones correspondientes a la frase mencionada.
\item	Fa8. El usuario sólo podrá decir frases cortas con duración no mayor a 10 segundos.
\end{itemize}

\item	Síntesis de voz a través del teclado

\begin{itemize}
\item	Fb1. El usuario a través del teclado en pantalla ingresa la frase a comunicar.
\item	Fb2. Haciendo uso de la API de Google de text to speech se realizará la síntesis del texto introducido.
\item	Fb3. El idioma soportado será el español de México.
\end{itemize}

\item	Diccionario del LSM

\begin{itemize}
\item	Fc1. El diccionario estará dividido por categorías.
\item	Fc2. Contará con filtros por categoría y un cuadro de búsqueda.
\item	Fc3. Al seleccionar la palabra, la aplicación le mostrará la imagen correspondiente, además de una breve descripción de su uso.
\end{itemize}

\end{enumerate}

Estas secciones serán alcanzables gracias a un menú lateral que nos ofrecerá la opción de ingresar a cada una.

\paragraph{Requisitos de rendimiento}
\begin{itemize}
\item	Rr1. La aplicación sólo podrá ejecutar una actividad a la vez, y será usada por un usuario.
\item	Rr2. La aplicación se conectará con el servidor que ejecutará el procesamiento de la voz.
\item	Rr3. El servidor tendrá la capacidad de atender a más de un usuario a la vez .
\item	Rr4. Se requiere que la aplicación cuente con conexión a Internet.
\item	Rr5. Dado el listado de palabras se plantea que la cantidad de registros almacenados en la base de datos sea en el orden de las decenas.
\end{itemize}

\paragraph{Restricciones de diseño}
\begin{itemize}
\item	Rd1. La interfaz de la aplicación móvil se desarrollará en Android Studio, haciendo uso de las formas ya establecidas.
\item	Rd2. La interfaz gráfica de la aplicación móvil se llevará a cabo siguiendo el diseño visual y de movimientos de la guía de Material Design para Android.
\item	Rd3. La interfaz será de fácil entendimiento para que todas las personas que lo deseen la puedan ocupar.
\item	Rd4. El servidor no contará con interfaz gráfica.
\item	Rd5. Se hará uso del diccionario “manos con voz” de María Serafín y Raúl González para la obtención de imágenes.
\end{itemize}

\paragraph{Atributos de la aplicación}
\begin{itemize}
\item	Aa1. La aplicación será diseñada para que pueda ser usada por todo tipo de personas.
\item	Aa2. No se requiere de un inicio de sesión, por lo que en un móvil pueden interactuar más personas.
\item	Aa3. Como trabajo a futuro serán ofrecidas las actualizaciones que permitan más características, así como también nuevas funciones.
\end{itemize}
