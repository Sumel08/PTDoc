\section{Marco teórico}

\subsection {Manejo de información personal}

Ya que la plataforma propuesta requerirá de un registro de usuarios para ligar a su boleto de acceso, se debe profundizar en el qué y cómo hacer una correcta manipulación de la información personal que resulte necesaria para el correcto funcionamiento de la plataforma.

El artículo 6 de la Ley Federal de Protección de Datos Personales en Posesión de los Particulares\cite{diarioOficial2010} dice “Los responsables en el tratamiento de datos personales, deberán observar los principios de licitud, consentimiento, información, calidad, finalidad, lealtad, proporcionalidad y responsabilidad, previstos en la Ley”. También cita en el índice III del artículo 10 que no será necesario el consentimiento para el tratamiento de los datos personales cuando los datos figuren en fuentes de acceso público.

Debido a que los únicos datos contemplados para el registro personal son el nombre, la edad y dirección de correo electrónico, todos ellos disponibles en fuentes de acceso público y obtenidos de manera clara y consensuada, no es necesaria la incorporación de ningún reglamento extraordinario para su manejo.

\subsection{API REST}

\subsubsection*{Concepto de API}

Una API es el mecanismo más útil para conectar dos softwares entre sí para el intercambio de mensajes o datos en formato estándar como XML o JSON. Así es como se convierte en un instrumento para buscar ingresos, abrirse al talento, innovar y automatizar procesos\cite{bancomer2016}.

Igual que una interfaz de usuario permite la interacción y comunicación entre un softwarey una persona, una API (acrónimo de Application Programming Interface) facilita la relación entre dos aplicaciones para el intercambio de mensajes o datos. Un conjunto de funciones y procedimientos que ofrece una biblioteca para que otro software la utilice como capa de abstracción, un espacio de acceso e intercambio de información adicional en la parte superior. Así una se sirve de la información de la otra sin dejar de ser independientes.

Cada API está diseñada en un lenguaje de programación concreto y con unas especificaciones distintas que la definen (las APIs pueden incluir especificaciones para estructuras de datos y rutinas, clases de objetos o
10variables, a partir de las cuales se basa el uso de esa interfaz). Además, suele ser habitual que cada una de ellas disponga de documentación completa y eficaz (un conjunto de tutoriales, manuales y reglas de buenas prácticas para esa interfaz de programación).

\subsubsection*{Tipos de API}

\begin{itemize}
  \item APIs de servicios web: son las interfaces de desarrollo de aplicaciones que permiten el intercambio de información entre un servicio web (software que da acceso a un servicio concreto a través de una URL) y una aplicación. Normalmente ese intercambio se produce a través de peticiones HTTP o HTTPS (la versión cifrada del protocolo HTTP). En la petición de la aplicación y respuesta, también en HTTP del servicio web, se contiene información de todo tipo tanto en los metadatos de la cabecera como en los del mensaje, normalmente en dos tipos de formatos muy usados: XML o JSON. Hay cuatro tipos de API de servicios web habituales entre los desarrolladores: SOAP (\textit{Simple Object Access Protocol}), un protocolo estándar de intercambio de información y datos en XML entre dos objetos; XML-RPC, un protocolo de llamada a procedimiento remoto que usa XML como formato de datos y llamadas HTTP como sistema de comunicación; JSON-RPC, mismo protocolo pero en formato JSON; y REST (\textit{Representational State Transfer}), arquitectura de software para sistemas hipermedia en la World Wide Web; una API REST usa el protocolo HTTP.
  \item APIs basadas en bibliotecas: este tipo de APIs son las que permiten que una aplicación importe una biblioteca de otro software para hacer el intercambio de información. Hoy en día gran parte de las bibliotecas que dan acceso a productos y servicios están diseñadas en JavaScript. Las APIs en JavaScript suelen ser un ejemplo ilustrativo de APIs basadas en bibliotecas, por ejemplo las que se utilizan dentro del mercado de la cartografía web (servicios como Google Maps, Leaflet, ArcGIS, CartoDB, MapBox o D3.js).
  \item APIs basadas en clases: este tipo de interfaces de desarrollo de aplicaciones permite la conexión con los datos en torno a las clases, como es habitual en programación orientada a objetos con Java. La API de Java usa clases abstractas para la creación de aplicaciones igual que cualquier programa desarrollado en este lenguaje. Esas clases proporcionan todo lo necesario para realizar todo tipo de funciones dentro
11de esas aplicaciones. La interfaz de desarrollo de Java se organiza en paquetes y cada uno de esos paquetes contiene a su vez un conjunto de clases relacionadas entre sí.
  \item APIs de funciones en sistemas operativos: los programas de software están continuamente interactuando con los sistemas operativos. Eso es una afirmación obvia. La realidad es que, en muchos casos, la forma en la que lo hacen es a través de APIs. Sistemas operativos como Windows disponen de APIs que permiten esa comunicación entre programas y el OS. Esta es la lista completa de API de Windows: interfaz de usuario, acceso y almacenamiento de datos, mensajería, gráficos y multimedia, diagnóstico de errores, etc.
  \item
  \item
\end{itemize}
