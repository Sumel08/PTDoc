\pagenumbering{arabic}
\setcounter{page}{10}
\section{Introducción}

En la sociedad moderna, distintos tipos de eventos que requieren o convocan a una gran asistencia tienen lugar en distintas fechas del año, ya sea con motivos lúdicos, culturales, políticos, económicos o informativos. El acceso a este tipo de eventos se gestiona de manera diferente; comúnmente por medio de boletaje que es vendido y gestionado a través de distintas plataformas tanto digitales como manuales.

Para llevar un registro y control de los accesos en estos eventos, diversos sistemas (en su mayoría digitales) se han desplegado, normalmente ligados a los registros de venta del boletaje. Por lo que es necesario que, para garantizar el acceso a las distintas zonas del evento, cada asistente lleve consigo el boleto adquirido como parte del proceso de registro. Estos sistemas conllevan una serie de desventajas, entre ellas que no están unificados (es decir, el sistema de control de asistencia puede variar con respecto de un evento al otro) y que el boleto no está propiamente ligado a una persona, sino a un código, y el proceso de pérdida y reemplazo es complicado.

Una gran vertiente de estos eventos de asistencia masiva son los festivales de música, en los que diversas agrupaciones o solistas se presentan para llevar a cabo espectáculos y presentaciones en uno o más escenarios colocados en una plaza que permita una gran afluencia, que se estima con base en la convocatoria que tengan cada uno de los artistas que se presentan en el festival. Como una incorporación exitosa a este modelo de negocio, se ofertan ventas al por menor a los asistentes de dichos eventos, ya sea de mercancía, servicios o consumibles.

Debido al número tan amplio de asistentes que se presentan en estos festivales, la organización de transacciones e interacciones que incluyan a dichos asistentes es un proceso complejo y tardado. Esto lleva a inconformidad tanto para los asistentes como para los organizadores del evento ya que, sin un método automatizado, resulta complicado llevar un registro y control de las eventualidades y capacidades que les sean concedidas.

Con la popularización de los festivales de música, las ventas al menudeo y la necesidad de hacer más eficiente el registro y concesiones que un evento de esta índole requiere, diversas plataformas se han desarrollado para atacar los problemas organizativos de las transacciones
2 (ventas) y asistencia, tanto por separado como de manera conjunta.
