\section{Planteamiento del problema}

En la actualidad, el concepto de festival de música se ha vuelto muy popular. En este tipo de eventos, la gente asiste para escuchar a distintas agrupaciones o solistas musicales comúnmente agrupados dentro de un género musical o nacionalidad.

Aunque gran parte del ingreso monetario que suponen los festivales de música provienen de la venta de boletos de acceso, de acuerdo con Expansión CNN, la mayoría de dichos ingresos se dan debido a las ventas al por menor que se hacen dentro del mismo evento\cite{Locker2013}. Aunque las ventas de los productos que se ofrecen en estos festivales no representan un ingreso directo para los organizadores del festival, sino que son un ingreso para vendedores que se registran en el evento y acceden a éste mediante una licitación, los montos obtenidos por el licenciamiento de dichas licitaciones son los que representan un ingreso mayoritario en dichos festivales.

A pesar de las ventajas que implica la venta de productos al por menor tanto para vendedores como para asistentes, las incomodidades y desorganización debido a la gran asistencia de gente deriva en grandes tiempos de espera, desorganización al momento del cobro, pérdida del control de inventarios y poco control del efectivo. Para procurar dar solución a los problemas que presenta el vendedor, constantemente se recurre a la implementación de plataformas propias de administración de inventarios y personal, sin embargo, los problemas que aquejan a los asistentes al evento quedan desatendidos con dichas soluciones.

Además, como alternativa para dar control de acceso, de acuerdo al boletaje adquirido por cada asistente, se recurre a una plataforma no integrada a las demás con este objetivo.

Derivado de las problemáticas mencionadas anteriormente surge la necesidad de desarrollar e implementar un sistema que unifique a las plataformas implementadas para solventar los problemas individuales y dar control e información a los administradores del evento. Esto deriva en la formulación de la siguiente pregunta:

¿Es necesario el desarrollo e implementación de una plataforma que permita el registro, control de acceso, manejo de monedero y de inventario en un evento de asistencia masiva?
